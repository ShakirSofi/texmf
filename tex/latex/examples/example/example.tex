\documentclass{hw}
\usepackage{code}

\title{Example \LaTeX{} Document} % defaults to empty string
\author{John Doe}                 % defaults to Michael Whittaker
\date{\today}                     % defaults to \today 

\begin{document}
\maketitle

\section{Math}
\[\cbox{
  \text{Vol}(d) = r^d \frac{2 \pi^{\frac{d}{2}}}{d \Gamma\group{\frac{d}{2}}}
}\]


\section{Code}
Refer to \listref{eqn-solver}, \listref{herb-sutter}, and
\listref{fast-inv-sqrt} for block code in various languages. Once \python{in} a
while, it is also nice to \c{struct}ure your text to have inline code in a wide
\cpp{class} of languages.

\begin{Python}[%
  label=list:eqn-solver,
  caption=Linear equation solver from ActiveState
]
def solve(eq,var='x'):
    eq1 = eq.replace("=","-(")+")"
    c = eval(eq1,{var:1j})
    return -c.real/c.imag
\end{Python}

\begin{CPP}[%
  label=list:herb-sutter,
  caption=Herb Sutter's favorite ten-liner
]
shared_ptr<widget> get_widget(int id) {
    static map<int, weak_ptr<widget>> cache;
    static mutex m;
    lock_guard<mutex> hold(m);
    auto sp = cache[id].lock();
    if (!sp) cache[id] = sp = load_widget(id);
    return sp;
}
\end{CPP}

\begin{C}[%
  label=list:fast-inv-sqrt, 
  caption=Fast inverse squareroot 
]
float Q_rsqrt( float number ) {
    long i; 
    float x2, y;
    const float threehalfs = 1.5F;
    x2 = number * 0.5F;
    y  = number;
    i  = * ( long * ) &y;                       // evil floating point bit level hacking
    i  = 0x5f3759df - ( i >> 1 );               // what the?
    y  = * ( float * ) &i;
    y  = y * ( threehalfs - ( x2 * y * y ) );   // 1st iteration
//  y  = y * ( threehalfs - ( x2 * y * y ) );   // 2nd iteration, this can be removed
 
    return y;
}  
\end{C}
\end{document}
